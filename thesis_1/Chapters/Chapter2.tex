% Chapter 3

\chapter{Case Study} % Main chapter title

\label{Chapter2} % For referencing the chapter elsewhere, use \ref{Chapter1} 

\lhead{Chapter 2. \emph{Case Studies}} % This is for the header on each page - perhaps a shortened title

%----------------------------------------------------------------------------------------

\section{General Design Considerations of Senior Population}
Senior center is an active node in the community that supply resources
to senior citizens and provide the community with aging related
knowledge~\cite{Dalsanto2009}. Main services offered at a senior
center include: education on a broad range of topics including health,
art, humanity, nutrition etc., volunteering opportunities,
intergeneration programs, meal plans, health screening, physical
training etc. 

There are several main categories of housing choices for the elderly:
independent living communities, assisted living facilities, Continuing
Care Retirement Communities (CCRC), nursing homes, and special care
facilities as Alzheimer's care facilities. The main difference are the
degree of care provided. Residents of independent living communities
differ from normal communities mainly in the demographic sence,
i.e. the residents are limited to senior citizens. Assisted living had
24 hour staff and provide living services such as meal, laundry and
bathing. It is meant for seniors that are not capable of living
independently but are not in need of heavy medical care. Nursing home
are more like hospitals with on site physicians and nurses that
provide high degree of medical care in addiction to living
services. CCRC provides varied degrees of service that covers all the
services provided in the housing types above and may include some
dementia care~\cite{JFCS2015}. Due to the awareness of the negative
impact of relocation of seniors especially those with dementia, the
CCRC prototype for senior living is the most suitable in the current
case.

The senior community center under discussion in the current project is
a combined community center and housing for senior citizens. It also
integrates with the university population by providing common space to
the community including university population, some housing units for
newly enrolled faculty members and space for elderly-children common
activities with the children from the children's school or from the
community. These makes the function different from a traditional
senior center setting. The case study in this section focus more on
the aspect specific to the project, such as the instances with mixed
age groups, affiliated to a university, or a combined facility of
living and research.

\section{Elderly and Children Combined Facility}
\subsection{Multi-generational Neighborhood Center}
In Europe, ``multi-generational neighborhood centers''
are alternatives to traditional senior centers that creates
inter-generational connections. The services expanded from traditional
senior center to include pre-school and infant care etc~\cite{Fromm2015}. 

\begin{comment}
\subsubsection{Intergenerational Quartier (Germany)}
The IGZ Quartier urban regeneration project is located in the city of
D{\"u}lmen. The Intergenerational Center of the project aims at ~\cite{Dulmen2014}
\end{comment}
\subsection{Mixed-generation Elderly Housing}
In Swabia, Germany, a housing program for elderly was created with 2/3
elderly residents and 1/3 of other age groups. The housing model aims
at helping elderly age in place with helps from other generations in a
``supportive environment''. The common space is extensible and can
hold a variety of activities arranged by social workers and the
residents themselves. The activites take place in the common space
include: morning play of children, affordable lunch for both the
senior residents and people from the neighborhood, informal community
gatherings and rent out space for other community events~\cite{Fromm2015}.

\subsection{Elderly Housing Next to Kindergarten}
\subsubsection{Altersheim Furttal, A Retirement Home in a Swiss Village}
The retirement home is built near the city center with good public
transportation. This connection provides the residents with a stronger
connection to the society.

There is a kindergarten to the north of the facility. The connections
between the two age groups are established with a common courtyard
between the kindergarten and the retirement home. The interior space
design strengthens this connection by arranging a two story ``lounge
space'' adjacent to the common garden.

\begin{figure}
\centering
\begin{subfigure}{0.7\textwidth}
  \centering
  \includegraphics[width=\linewidth]{case1.png}
  \caption{Site Plan Layout of Altersheim Furttal and Kindergarten}
  \label{fig:case1}
\end{subfigure}
\begin{subfigure}{0.7\textwidth}
  \centering
  \includegraphics[width=\linewidth]{case2.png}
  \caption{Views of the Lounge Area}
  \label{fig:case2}
\end{subfigure}
\caption{Common Garden and Interior Design in Creating Connections
  between Different Age Groups}
\label{fig:case2}
\end{figure}
\section{University Affiliated Senior Housing}
\subsection{Ithaca College and Longview Partnership}
The Longview project is a combined effort of Ithacca College, Cornell
University and City of Ithaca. It started as an renovation of Tompkins
County Hospital and grow to a CCRC facility providing a variety of
housing choices include independent living, assisted living and
enhanced assisted living.

\subsubsection{Connections between Longview and Ithaca College}
The connections between the Longview program and Ithaca College
include: education opportunities, access to school facilities,
volunteering opportunities and therapy programs supported by students
and staff from related medical programs.

Residents are provided with education opportunities: they have access
to classes taught at Ithaca College or in the Ithaca classroom in
Longview. School facilities such as libraries and gyms are open to
residents to use. School recreational activities are also open for
Longview residents such as sports, art and music events. Students
volunteers participate in the activity arrangement of the Longview
program. Students in the Colloege Physical Therapy and Occupational
Therapy help the staff members give physical traingings to elderly
residents at Longview. This collaboration benefits both the college
students and the staff and elderly residents at Longview. Students
gain parctice experiences and staff and elderlies gain knowledge and
skills for maintaining good physical conditions.

\subsubsection{Site Plan and Building Layout}
Longview community is located to the southwestern of the main campus
of Ithaca College~\cite{googleMapLongiew}.
\begin{figure}[htbp]
  \centering
  \includegraphics[width=0.7\linewidth]{siteLongview.png}
  \caption[Site, Longview]{Site Longview~\cite{googleMapLongiew}}
  \label{fig:siteLongview}
\end{figure}

\subsubsection{Activities}
Activities in Longview includes physical activities, gardening,
Intergenerational Choir singing etc. As a result of the collaboration
with Ithaca College, residents can take courses and have access with
the college facilities such as library, gym and art
performances. Recreational facilities in Longview include craft room,
fitness room, game room, green house, library, hair salon, massage
room and walking trail landscape design on the west of the facility.

\subsubsection{Housing Choices and Unit Design}
There are 100 units of independent living appartment units. The unit
types of independent living include small studio, One-bedroom and
two-bedroom unit. All three types of living units include kitchen,
bathroom and a balcony. The studio has a combined living room and
bedroom while the other two types have separate living room and
bedroom. The size of studio, single bedroom and large bedroom are 465
sq.ft., 600 sq.ft. and 858 sq.ft. ~\cite{LongviewIndepend}. 
\begin{figure}[htbp]
  \centering
  \includegraphics[width=0.7\linewidth]{independUnit.png}
  \caption[Independent Living Unit, Longview]{Independent Living
    Unit, Longview~\cite{LongviewIndepend}}
  \label{fig:independUnit}
\end{figure}

There are also 22 Independent living Patio Homes to the west of the
main longview main appartment building providing high level of living
quality. The Patio unit has a total area of 1355sq.ft. (without
garage)~\cite{LongviewPatio}. The living unit include two bedrooms ,
two bathrooms, one with shower and the other with bathtub, a living
room, a kitchen, a laundry room, a garage and a patio.
\begin{figure}[htbp]
  \centering
  \includegraphics[width=0.7\linewidth]{patioLongview.png}
  \caption[Patio Living Unit, Longview]{Patio Living
    Unit, Longview~\cite{patioLongview}}
  \label{fig:patioLongview}
\end{figure}

There are 60 assisted living units, providing 24-hour assistance, no
site nurse and emergency pull cords. The assisted living unit is
250sq.ft., smaller than the independent living units. It inlucdes a
large with a bath room, a combined living and sleeping area and a
small refridgerator. No kitchen or balcony is included in the assisted
living units.
\begin{figure}[htbp]
  \centering
  \includegraphics[width=0.7\linewidth]{assistLongview.png}
  \caption[Independent Living Unit, Longview]{Independent Living
    Unit, Longview~\cite{LongviewAssist}}
  \label{fig:assistLongview}
\end{figure}
There are also 24 enhanced assisted living units with additional cares
for residents with memory problems. These units located on the
``Garden Level'' with secure system that monitors exits. Residents
and families also have the option to wear a bracelet sensor for closer
monitoring. The enhanced assisted living units is 220 sq. ft. with the
same function layout as the assisted living units.
\begin{figure}[htbp]
  \centering
  \includegraphics[width=0.7\linewidth]{enhanceLongview.png}
  \caption[Independent Living Unit, Longview]{Independent Living
    Unit, Longview~\cite{enhanceLongview}}
  \label{fig:enhanceLongview}
\end{figure}
\section{Dementia Assisted Living}
``Dementia is an umbrella term for a group of cognitive disorders
typically characterized by memory impairment, as well as marked
difficulty in the domains of language, motor activity, object
recognition, and disturbance of executive function – the ability to
plan, organize, and abstract.''~\cite{CDCdementia} Dementia, or its
most common form Alzheimer is highly prone and one could not neglect
its existance: there are 5 million Alzheimer victims in the U.S. and
every 1 out of 3 seniors die in dementia. Women are more vulnerable to
dementia and 2/3 of the Alzheimer victims are
women~\cite{alzorg2014}. This section conduct some related case study
on elderly caring facilities for people with Alzheimer Diseases. 

The physical space acted as a ``therapeutic resource'' in improving
the wellbeing and help reduce the seriousness of
dementia~\cite{Day2000}. Relocation of individual dementia victims to
new environments can increase the possibility of depression and
mortality~\cite{ANTHONY01111987}. This implies the necessity for
dementia dedicated space. If there are not such spaces, when resident
develops dementia, they will have to be relocated to facilities that
has dementia care functions, which might cast negative impact. The
living unit for cognitively impared people are commonly refered to as
Spetial Care Unit (SCU). The common features of SCUs include ``smaller
size units, fewer resident rooms and more designated private rooms
with private dining rooms''. The SCU environment have positive impact
on ``communication, self-care, social function and mobility'' status
of dementia victims. It also reduce emotional strain and increase
satisfactions. Separation between people with and without dementia is
necessary as study showed non-dementia residents experience mental
declines as a result of living near dementia victims. The tipical
features of a SCU unit include: less rooms, small room sizes, private
rooms and dining space, access to outdoor environment
etc~\cite{Day2000}. Smaller cluster size showed positive effects on
reducing agitation level, aggressiveness, anxiety and
depression~\cite{Day2000}. The positive impact of smaller cluster
group setting also includes relief of stress and negative attitude of
relative care-givers~\cite{Annerstedt19931529, Day2000}. Special
acoustic feature should be added to create a balance between ``sensory
overstimulation'' and ``deprivation'', i.e. create a space that is
neither too noisy nor too quiet. Since people with dimentia tend to
also have visual difficulties, the suggested visual environment is low
glare, high contrast and increase lighting level~\cite{Day2000}. The
``bright light treatment'' showed improvement on sleep
patterns~\cite{Mishima1994}. For enhancing way-finding, common design
suggestions include: provide views to the outside environment which
gives hint of location; create ``landmarks'', large signs; increase
lighting level of public spaces etc. Corridors are associated with
less orientation and higher percentage of hallway reduces
disorientation~\cite{Day2000}
\section{Sustainable Strategy in Senior Center}